\chapter{Post-Processing Techniques for Instance Segmentation Masks}\label{ch:postProcessing}

At this stage, the system is capable of detecting cetaceans at an individual pixel level. Before these detections can be passed to the identification module, some post processing of the output must be performed to allow for both the reduction in computational expense of operating on the detector's output as well as ensuring that no potentially important information which will assist in an identification is lost. 


\section{Background Subtraction \& Cropping}\label{ch:postProcessing,sec:bgExtraction}

One of the main components of the post-processing pipeline is the background subtraction module. This is an extremely important step in ensuring an accurate individual classification based on the detected fin. By removing the surrounding background from around the fin, this reduces the amount of noise the identifier will be required to deal with. Thanks to the segmentation masks produced by the Mask-RCNN detector \cite{he_mask_2017} in Chapter \ref{ch:cetDet}, the pixels which are likely background have been identified.

As such, it is possible to utilise this mask to perform background subtraction. As both the input image and its resultant mask can be represented as matrices, these can be manipulated such that if pixel$_{i, j}$ in the input image is denoted as background in the mask, the values of pixel$_{i, j}$ can be set to [255, 255, 255] (white). This has the effect of whiting out any pixels not detected as part of the fin in the image, removing noise. 

Once the background subtraction has been achieved the input image can then be cropped to reduce the image down as much as possible. By using the top, bottom, left, and right-most non-white pixels in the image as bounding box coordinates for the fin, the input image can be reduced down greatly, often to only a few hundred pixels in both height and width. This greatly reduces the computational expense of further operations downstream by reducing the size of subsequent input images passed to other components. 

Figure \ref{fig:fin-extraction-clean} shows the effect of performing background subtraction on an input image, with the detected \texttt{dolphin} pixels from the Mask-RCNN highlighted in red. As can be seen, the background subtraction and cropping has resulted in a clean image of the animal's dorsal fin. Identifying information is present, with minimal levels of noise. 

\begin{figure}[h]
	\begin{center}
		\includegraphics[scale=0.5]{Chapter4/figs/fin-extraction-clean.png}
	\end{center}
	\caption{The effect of background subtraction and cropping on an input image, left. The detected \texttt{dolphin} pixels by the Mask-RCNN have been highlighted in red with confidence score shown. The resultant output, right, has been enlarged for visibility.}
	\label{fig:fin-extraction-clean}
\end{figure}

Whilst the background subtraction module aims to reduce as much noise as possible entering the identification module, which can be achieved thanks to the high accuracy of the detector, it will not be possible to remove all noise. It may be the case, such as in Figure \ref{fig:fin-extraction-unclean}, whereby some background has been mislabelled as \texttt{dolphin}. As a result, the background subtraction module is unable to remove the mislabelled background pixels which may effect the accuracy of the identification downstream. 

\begin{figure}[h]
	\begin{center}
		\includegraphics[scale=0.5]{Chapter4/figs/fin-extraction-unclean.png}
	\end{center}
	\caption{The result of background subtraction and cropping where the detector has mislabelled some background pixels as \texttt{dolphin}, right. Detected pixels have been highlighted red on the input image, left, with confidence score shown.}
	\label{fig:fin-extraction-unclean}
\end{figure}

%% FInd image of pod, show multiple crops.



%%%%%%%%%%%%%%%%%%%
\nomenclature[z-CNN]{CNN}{Convolutional Neural Networks}

