\chapter{North Sea Data Collection} \label{ch:NDD}

At this stage, the automatic photo-identification system in development is capable of detecting cetaceans in large panoramic images and post-processing these detections into a form ready for individual identification, as outlined in Chapter \ref{ch:cetDet}. So far, the detector has been trained and tested on data collected by Newcastle University's Marine MEGAfauna Lab whilst researching the abundance of Indo-Pacific bottlenose dolphins (\textit{Tursiops aduncus}) off the coast of Zanzibar, Tanzania \cite{sharpe_indian_2019}. 

Whilst the Marine MEGAfauna Lab research heavily in the Indian Ocean around Zanzibar \cite{yang_description_2020, temple_life-history_2020, temple_marine_2019, temple_marine_2018, weigmann_revision_2020, barrowclift_social_2017}, this is not the only place they operate \cite{temple_by-catch_2021, yang_classification_2017, yang_influence_2022}. In recent years, their work has begun to include more local waters such as the North Sea off the coast of Northumberland, UK \cite{van_bressem_visual_2018, yang_characterization_2021}. These waters are known to host a wide variety of marine mammals, with the Marine MEGAfauna Lab focussing efforts specifically on the bottlenose dolphin (\textit{Tursiops truncatus}) and white-beaked dolphin (\textit{Lagenorhynchus albirostris}) populations. 

Near the end of the Mask-RCNN detection model's completion, the Marine MEGAfauna Lab were preparing to begin a large scale bottlenose and white-beaked abundance estimate and health assessment survey. As such the opportunity arose to validate the detector's generalisability on data which is similar in composition and purpose, however was collected in a different geographic location, at a different time, and containing different species to the data used to train and test the detector.

As a result, this Chapter discusses the collection of abundance estimate data in Northumberland, UK for the purpose of model and technique evaluation as outlined in Chapter \ref{ch:cetDet}. In order to achieve this evaluation, the photo-identification data was curated and transformed from a biological catalogue into a computer vision dataset known as The Northumberland Dolphin Dataset 2020 (NDD20) - the creation of which will also be discussed. 

\section{Data Collection}\label{ch:NDD,sec:dataCollection}

The following Section provides context for the data collection survey, beginning by briefly outlining the geographic area in which the data was collected and discussion of why the area was chosen. Next the survey effort is discussed in detail, including a run-down of the methodology used, for the purposes of reproducibility. 

\subsection{The Study Area}\label{ch:NDD,sec:dataCollection,sub:studyArea}

Data collection was conducted in and around the Coquet to St. Mary's Marine Conservation Zone (MCZ), located off the coast of Northumberland, UK. The MCZ, established in January 2016 through powers granted by the Marine and Coastal Access Act 2009 \cite{noauthor_marine_2009}, covers approximately 40km of coastline from Almouth in the north to Whitley Bay in the south, extending outwards 7.5km at its greatest to cover an area around 192km$^{2}$. A map of the MCZ and its location with respect to the UK can be seen in Figure XXX.

%TODO: Figure of map with track lines

The area is of high importance, supporting a wide variety of marine life thanks to sections of intertidal and sub-tidal rock and sediment, making it fertile feeding grounds for the bottlenose and white-beaked dolphins which make use of the area. As a result of this fertility, as well as waters up to 30m deep in some places, the MCZ sees high levels of fishing activity - typically for crustaceans using pots \cite{stephenson_spatial_2017}. Whilst these fishing vessels operate from a number of small ports throughout the North East of England, the MCZ itself lies close to the large Port of Blyth. As a result, the MCZ boundary provides a 250m buffer zone around the limits of the port in order to reduce economic damage. This survey region was selected as no previous surveying had been undertaken in the area.

\subsection{Survey Effort}\label{ch:NDD,sec:dataCollection,sub:surveyArea}

Dedicated bottlenose and white-beaked dolphin photo-identification surveys were conducted in the MCZ between 19/07/2019 and 10/10/2019, with a total of 27 surveys undertaken. These were performed using a 5.6m rigid inflatable boat (RIB) with a 50 horsepower four-stroke outboard engine. All surveys began from Newcastle University's Blyth Marine Station, located in the Port of Blyth, before entering the MCZ.

Surveys initially began by following set transect lines, traversing between the northern and southern-most points of the MCZ. Thanks to limited success encountering individuals strictly following transect lines however, the survey switched to more opportunistic surveying based on reports from two citizen science groups: the Newbiggin-by-the-Sea Dolphin Watch\footnote{Newbiggin-by-the-Sea Dolphin Watch: \href{https://en-gb.facebook.com/groups/NEWILDDOLPHINMONITORINGPROJECT/}{facebook.com/groups/NEWILDDOLPHINMONITORINGPROJECT}} and the North East Cetacean Project\footnote{North East Cetacean Project: \href{https://en-gb.facebook.com/groups/NorthEastCetaceanProject/about/}{facebook.com/groups/NorthEastCetaceanProject}}. The use of citizen scientists for photo-id surveys has seen increased prevalence in recent years, with multiple studies producing promising results if access to groups of dedicated citizen scientists is available, like in Northumberland \cite{araujo_population_2017, currie_conservation_2018, armstrong_photographic_2019, araujo_photo-id_2019}. Track lines showing movement of the vessel were recorded via GPS tracking, and can be seen in Figure XXX. When dolphins were encountered, the time stamp was recorded alongside other effort data such as direction of travel, sea state, species, group size, and demographic composition.

Surveys were only conducted in Beaufort Sea States $\le 3$ \cite{world_meteorologicial_society_beaufort_1970} without heavy rain. Outside of these conditions surveying can become unsafe and the photographs unusable for photo-id due to swell and lens splash. Due to the nature of the North Sea, conditions outside of these restrictions can be common. Surveying was performed using the constant scanning method \cite{mann_behavioral_1999}, with cues including sight of dorsal fins breaching the waterline, splashing, and leaping. For each survey the vessel was manned by at least two dedicated observers and a skipper, in line with other photo-id surveys \cite{sharpe_indian_2019, bessesen_lacaziosis-like_2014, silva_winter_2012}.

Individual dolphins in an encounter were photographed randomly using a Canon EOS 550D Digital SLR with a Canon 70–200mm zoom lens, aiming to capture photographic data for every individual present. Multiple photographs were captured of each cetacean over the course of the encounter to ensure identifiable information could be fully captured. When capturing an encounter care was taken not to approach individual cetaceans at an angle less than $30^{\circ}$, keeping as parallel as possible and to speeds no greater than 6 knots in order to prevent the cetaceans becoming stressed or injured as per Marine Management Organisation guidelines. All members of the survey team were trained in minimising wildlife disturbance through the WiSe Scheme by the Yorkshire Wildlife Trust\footnote{WiSe Scheme: \href{https://www.wisescheme.org/}{wisescheme.org}}, with the survey itself having the approval of Newcastle University's Ethics Board.

\subsection{Field Season Summary}\label{ch:NDD,sec:dataCollection,sub:summary}

Of the 27 days where surveys were conducted, 14 contained encounters. Of these 12 were made up of bottlenose dolphin; only two were made up of white-beaked dolphin. No encounters contained both species. Groups were defined using the 10m chain rule \cite{smolker_sex_1992}. Group size averaged 12 for bottlenose dolphins, typical for the species \cite{shane_ecology_1986}. Altogether 40 individuals were identified and catalogued, broken down into 30 bottlenose and 10 white-beaked dolphins. Of all animals encountered, 27\% were calves. They have been excluded from this analysis as they could not be considered independent due to reliance on their mothers, and had not yet developed permanent markings.

A total of XXX images were captured throughout the fieldwork season. These were processed for use in the photo-identification catalogue as to remove any images with no value, such as those which were out of focus or did not contain any cetacean. Animals present in the images were coded according to their distinctiveness as per the guidelines presented in \cite{urian_recommendations_2015}. Those coded D1 were considered very distinctive with little chance of misidentification, whilst those coded D2 were considered moderately distinctive with small prominent markings which could allow for a high chance of correct classification provided the image is clear. CF coded individuals were those which contained little to no identifying information which have a high chance of misclassification. Once coded, animals considered D1 and D2 were individually identified resulting in a catalogue of XXX images. 

\section{The Northumberland Dolphin Dataset 2020}\label{ch:NDD,sec:NDD20}

The fieldwork season and data processing undertaken resulted in a photo-identification catalogue of bottlenose and white-beaked dolphins currently inhabiting the Coquet to St. Mary's MCZ. Photo-identification catalogues utilised in marine biology however are not in the form required for training or validating a computer vision model. As such, further processing was required to transform the catalogue into a dataset capable of both validating the instance segmentation model developed in Chapter \ref{ch:cetDet}, as well as training a model for fine-grained individual identification. This Section discusses the creation of the Northumberland Dolphin Dataset 2020 (NDD20) \cite{trotter_ndd20_2020}, the computer vision dataset created from the photo-identification catalogue.

\subsection{Data Pre-processing}\label{ch:NDD,sec:NDD20,sub:dataPreprocessing}

%TODO: Write up of the data preprocessing. Removed images unsuitable for object detection (make clear the dataset has multiple levels of coarseness). DSP presentation/NDD20 paper good here.

\subsection{Data Labelling}\label{ch:NDD,sec:NDD20,sub:dataLabelling}

%TODO: Discuss data labelling, JobsOC. Training provided, checked after for errors. Only object detection was labelled by JobsOC
%Used VIA \cite{dutta_via_2019} as with Zanzibar data.
% Species ID and Individual ID labelled by me.

\subsection{Underwater Data Addition}\label{ch:NDD,sec:NDD20,sub:underwaterData}

%TODO: Discuss the addition of the underwater data from the Farnes Deep. NDD20 paper good here. 

\subsection{NDD20 Summary}\label{ch:NDD,sec:NDD20,sub:NDD20Summary}

%TODO: Summary of NDD20. Take from NDD20 paper? Num imgs, classes, etc.
% Basline experiments from NDD20 paper.
% Link to access NDD20. Use in Kaggle competition.

%%%%%%%%%%%%%%%%%%%
\nomenclature[z-MCZ]{MCZ}{Marine Conservation Zone}
\nomenclature[z-RIB]{RIB}{Rigid Inflatable Boat}
\nomenclature[z-NDD20]{NDD20}{Northumberland Dolphin Dataset 2020}

